\chapter{Podsumowanie i~wnioski}\label{chap:podsumowanie}

\section{Dyskusja wyników}

Uzyskane wyniki są gorsze od tych ze zbadanych prac naukowych, nie licząc kilku przypadków.

Model \shortcut{GMM-UBM}, prawdopodobnie dzięki jego prostocie, udało się odtworzyć i może pełnić
rolę \foreign{baseline} w pracy. Sposób weryfikacji tekstu przez odpowiednie zaadaptowanie \shortcut{GMM}
okazał się skuteczny, choć model ten nie uwzględnia tego jak fonemy następują po sobie w czasie.

Niestety systemu \shortcut{HMM-GMM} nie udało się wytrenować. Możliwe, że należało podobnie jak w przypadku \shortcut{DNN}
użyć gotowego podsystemu do rozpoznawania mowy.

Stworzony model \shortcut{DNN-GMM} jest istotny, gdyż nie jest wynikiem naśladowania innej pracy, lecz jest w pewnym
stopniu nowy. Jego skuteczność niestety cierpi z powodu tego, że \techname{DeepSpeech} wykrywa litery, choć wygodniejsze byłyby
fonemy. Nie jest jednak możliwe wytrenowanie go od podstaw przy obecnej dostępności zasobów i zbiorów danych.
Również ocena zgodności mówców mogłaby zyskać na skuteczności, gdyby wykorzystać popularne i-wektory. W problemie weryfikacji
treści model ten radzi sobie najlepiej i ma tutaj dodatkową zaletę, iż można go użyć do weryfikacji dowolnej treści, nie
jest ograniczony do zamkniętego zbioru zdań jak \shortcut{GMM-UBM}. Ostatecznie wyniki są wyraźnie odmienne od losowych.

\section{Perspektywy dalszych badań}

Wyniki tworzonych modeli z pewnością dałoby się poprawić wykonując dokładną
optymalizację hiperparametrów. Podobnie można spróbować zdobyć inny zbiór danych
niż \techname{RedDots} na potrzeby treningu i walidacji. Tylko wtedy możliwe
byłoby porównanie wyników z tej pracy z wynikami innych prac.

Metoda wyznaczania prawdopodobieństwo przynależności przez obliczanie stosunku
przynależności do mikstur gaussowskich użytkownika oraz modelu tła nie jest
tak powszechnie stosowana jak metoda i-wektorów. Połączenie tej metody
z metodą przypisywania ramek z tej pracy mogłoby poprawić wyniki.
Zwłaszcza, że i-wektory można wyliczyć w oparciu o statystyki z głębokich sieci
neuronowych do rozpoznawania fonemów. W przejrzanych pracach te sieci bazowały
na przypisaniu ramek dokonanym przez system \shortcut{HMM-GMM}.

