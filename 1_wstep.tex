\chapter{Wstęp}\label{chap:wstep}

Zakresem niniejszej pracy magisterskiej są systemy rozpoznawania mówcy zależne od tekstu. Celem
pracy jest zbadanie możliwości wykorzystania modelu sieci neuronowej do rozwiązania tego problemu. 

Rozpoznawanie mówcy polega na analizie nagrań dźwiękowych mowy w celu wydobycia cech biometrycznych 
charakteryzujących osobę mówiącą. Wyróżnia się dwa praktyczne problemy do rozwiązania z wykorzystaniem tych
cech: problem weryfikacji mówcy oraz problem identyfikacji mówcy. 

Weryfikacja mówcy polega na stwierdzeniu, czy nagranie pochodzi od pewnej zarejestrowanej wcześniej osoby 
lub wykryciu, że jest na nim ktoś inny. Ten przypadek występuje na przykład przy uwierzytelnianiu za pomocą głosu 
do systemu bankowego. Konieczne jest wtedy wykrycie, gdy oszust podaje się za prawdziwego właściciela konta. 
Identyfikacja mówcy polega na rozpoznaniu, która z zarejestrowanych osób jest na nagraniu lub czy 
jest na nim ktoś niezarejestrowany. Identyfikacja może być wykorzystana przez służby policyjne do
zidentyfikowania osób na zdobytym nagraniu.

Rozpoznawanie mówcy można również podzielić ze względu na to, czy treść nagrania jest znana z góry
na rozpoznawanie mówcy zależne od tekstu i niezależne od tekstu. Weryfikacja mówcy zwykle może
być przeprowadzona zależnie od tekstu, gdyż w typowych zastosowaniach można oczekiwać od weryfikowanej
osoby kooperacji. Znajomość treści jest cenna, gdyż pozwala zwiększyć skuteczność, szczególnie w sytuacji, 
gdy nagrań rejestrujących jest tylko kilka. W przypadku, gdy nagranie nie zawiera oczekiwanej treści 
akceptowalne jest odrzucenie weryfikowanego nagrania niezależnie od mówcy. 
Identyfikację z kolei trzeba przeprowadzać bez założeń co do treści nagrania, gdyż nagrywane osoby mogą 
być tego nieświadome.
% cite że poprawia skuteczność jak mało nagrań

Jako że przedmiotem pracy jest rozpoznawanie mówcy zależne od tekstu, uwaga zostanie poświęcona problemowi
weryfikacji, a nie identyfikacji mówcy. Ten problem jest również ciekawy ze względu na potencjalne zastosowania,
przypuszczalnie większe niż problem identyfikacji. Zaletą bazowania na nagraniu mowy jest możliwość
jego nieinwazyjnego pozyskania, jednakże niska skuteczność rozpoznawania sprawiała, że takie systemy
mogły pełnić tylko rolę drugiej warstwy zabezpieczeń, jako uzupełnienie tradycyjnego hasła.

Systemu bazujące na sieciach neuronowych osiągnęły w ostatnim czasie znakomite wyniki w problemach
związanych z przetwarzaniem obrazów, dźwięków czy dokumentów tekstowych. Rekurencyjne sieci neuronowe
stanowią u podstaw najnowszych systemów do rozpoznawania mowy. Można się spodziewać, iż mogą one
osiągnąć również dobre wyniki przy rozpoznawaniu mówcy. Rzeczywiście, miały już miejsce próby
wykorzystania głębokich sieci neuronowych np. przez zastąpienie nimi \shortcut{GMM} w obecnych systemach do rozpoznania 
jaki fonem znajduje się w ramce nagrania i wydobycia z nich tzw. \foreign{bottleneck features}.
Korporacje jak Google również zgromadziły zbiory danych i zaproponowały architekturę bazującą na sieciach
neuronowych. Celem tej pracy jest zbadanie tematu i być może wypróbowanie innego sposobu zaaplikowania
sieci neuronowej.
% bottleneck paper, google paper

Zainteresowanie tym tematem i potencjalne zyski, które mogą z niego płynąć, sprawiło niestety, 
że zbiory danych dopasowane do badania tego problemu są albo chronione przez przedsiębiorców, 
albo dostępne, lecz za opłatą. Na szczęście w 2015 ta potrzeba
została zaadresowana przez naukowców między innymi z Singapuru. Zgromadzili oni nagrania o znanej
treści od ochotników z całego świata i opublikowali zbiór RedDots. Trzeba jednak przyznać, że
zasoby posiadane przez korporacje w postaci danych i w postaci czasu i talentu pracowników sprawiają,
że dorównanie ich wynikom jest nierealne. Tym niemniej możliwe jest zbadanie, czy wybrany pomysł
jest w ogóle warty uwagi i porównanie go z systemami \foreign{baseline}.

\section{Cele pracy}\label{sec:cele_pracy}

Celem pracy jest stworzenie systemu do rozpoznawania mówcy wykorzystującego sieć neuronową. 
Zostanie on przetestowany na zbiorze RedDots i porównany z innymi stworzonymi systemami.
Nie oczekujemy uzyskania lepszych wyników, lecz warto by było, gdyby świadczyły o tym,
że dana metoda nie jest zupełnie losowa.

Systemy, które zostały stworzone i przetestowane, to:

\begin{itemize}
    \item System bazujący na miksturach wielowymiarowych rozkładów normalnych i modelu tła (\shortcut{GMM-UBM})
    \item System wykorzystujący ukryty model Markowa z emisjami będącymi miksturami wielowymiarowych rozkładów normalnych (\shortcut{HMM-GMM})
    \item System wykorzystujący rekurencyjną sieć neuronową do dopasowania ramek do fonemów, a następnie modelujący charakterystyczny sposób generowania określonych fonemów przez mówców za pomocą mikstur wielowymiarowych rozkładów normalnych.
\end{itemize}

Modele zostaną przetestowane na zbiorze RedDots. Policzone zostaną następujące miary jakości:

\begin{itemize}
    \item Krzywa \shortcut{ROC}
    \item \shortcut{EER}
    \item \shortcut{AUR}
\end{itemize}

Modele zostaną na ich podstawie porównane między sobą i z wynikami innych prac. Osobno zostaną rozważeni mężczyźni 
i kobiety, gdyż zbiór zawiera o wiele nagrań mężczyzn i wiele prac zupełnie ignoruje wyniki dla kobiet, 
gdyż jest ich za mało. Rozpatrzone zostaną cztery zestawy testowe zdefiniowane w RedDots.

\section{Przegląd literatury}\label{sec:przeglad_literatury}

Podstawy teorii rozpoznawania mowy są dobrze opisane w \titlei{Fundamentals of speech recognition} 
od dr Lawrence Rabinera oraz w jego 
\titlei{A Tutorial on Hidden Markov Models and Selected Applications in Speech Recognition}, które
skupia się na praktycznym wprowadzeniu do ukrytych modeli Markowa, użytych w tej pracy.
Innym bardzo praktycznym zasobem na ten temat jest \titlei{The HTK Book}, opisująca rozwiązanie
tego problemu zastosowane w pakiecie \titlei{Hidden Markov Model Toolkit} i publicznie dostępne zasoby
kursu \titlei{CS 224S Spoken Language Processing} z Uniwersytetu Stanford.

O rozpoznawaniu mówcy traktuje \titlei{Fundamentals of Speaker Recognition} od prof. Homayoona Beigi.
Jest to przydatny zasób, jednakże bardziej pomocne może okazać przejrzenie się mniejszych, lecz
bardziej aktualnych publikacji na ten temat.
% cite Rabiner, Rabiner tutorial, Beigi

Ogólnie tematy związane z uczeniem maszynowym doskonale opisane są w \titlei{Pattern Classification} prof. Richarda Dudy.
Porusza takie tematy jak statystyka Bayesowska, \foreign{expectation maximization}, sieci neuronowe, RBM 
i \foreign{unsupervised learning}. Autor znalazł ten tekst jako przystępniejszy w odbiorze niż powszechnie 
uznany i polecany studentom \titlei{Pattern Recognition and Machine Learning} Christophera Bishopa. 
Doskonałym zasobem do głębokich sieci neuronowych jest dostępna legalnie w Internecie książka 
\titlei{Deep Learning}\cite{deeplearningbook} od Iana Godfellowa i innych.

Godne polecenia są kursy internetowe dostępne za darmo na platformie Coursera.
Wysoko oceniane, również przez autora, związane z tematem pracy kursy to
\titlei{Machine Learning} od Andrew Ng, 
\titlei{Neural Networks for Machine Learning} od Geoffrey Hintona,
\titlei{Digital Signal Processing} od Paolo Prandoniego i  Martina Vetterliego.
Są to zaadoptowane do formy kursu internetowego kursy akademickie. Pozwalają 
doświadczyć lekcji prowadzonych przez światowej sławy naukowców, niezależnie
od miejsca urodzenia, co jest bardzo cenne.

\section{Układ pracy}\label{sec:uklad_pracy}

Rozdział \ref{chap:wstep} zawiera wprowadzenie i określenie tematu oraz celu
pracy. Rozdział \ref{chap:teoria} zawiera opis kilku zagadnień z zakresu
rozpoznawania mówcy i rozpoznawania mowy, które zostały wybrane jako
istotne dla tej pracy. Rozdział \ref{chap:technologie}
opisuje zbiór danych, technologie i narzędzia wykorzystane w pracy. W rozdziale
\ref{chap:badania} przedstawiono opis stworzonych systemów i wyniki testów. Rozdział
\ref{chap:podsumowanie} zawiera porównanie wyników i podsumowanie czy założone 
cele zostały osiągnięte. W dodatku \ref{app:plyta} znajduje się płyta CD z kodem aplikacji,
wersją elektroniczną tej pracy oraz kopią wykorzystanych źródeł internetowych.

