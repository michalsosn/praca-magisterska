\chapter{Użyte narzędzia i~technologie}\label{chap:technologie}

\section{Zbiór danych}\label{sec:zbior_danych}

% RedDots only, CMUDict

\section{Sprzęt}\label{sec:sprzet}

\section{Technologie}\label{sec:technologie}

\subsection{Język programowania}

Glasgow Haskell Compiler został napisany w~języku Haskell. Wykorzystano została
w~projekcie technika \foreign{bootstraping}, polegająca na tym, iż kompilator
rozwijany jest w~języku, który kompiluje i~jedną z~zależności potrzebną do
zbudowania GHC jest on sam. Środowiskiem wykorzystanym do edycji kodu było
IntelliJ IDEA z~wtyczką zapewniającą wsparcie dla języka Haskell. Wtyczka ta
niestety w~pełni działa tylko z~projektami systemu Cabal i~nie jest dostosowana
do prac nad GHC. Z~tego powodu środowisko nie zapewniało na przykład ciągłego
raportowania błędów w~kodzie lub wyszukiwania deklaracji określonego typu, choć
z~tych udogodnień można korzystać w~innych projektach.

Testowanie i~walidacja kompilatora są realizowane przez skrypty w~języku
Python. Jest on wymagany do zbudowania GHC
również z~tego powodu, że do stworzenia dokumentacji wykorzystane zostało
narzędzie Sphinx, wymagające interpretera.

W~projekcie użyty został również język C do napisania środowiska
uruchomieniowego, linkowanego z~każdym programem skompilowanym przez GHC.
W~pozostałych częściach GHC można znaleźć dyrektywy preprocesora C, który GHC może
wykorzystać zanim przejdzie do kompilacji. W~tej pracy jednak kontakt z~nim był
znikomy.

\subsection{Biblioteki}

\subsection{Narzędzia}

% Jupyter, Docker, git?

