\chapter{Płyta CD}\label{app:plyta}

\begin{figure}[htb]
\makebox[\textwidth]{\framebox[12.8cm]{\rule{0pt}{12.8cm}}}
\end{figure}
\pagebreak

Zawartość katalogów na płycie:
\begin{description}
    \item[doc] : elektroniczna wersja pracy dyplomowej oraz dwie prezentacje wygłoszone podczas seminarium dyplomowego
    \item[src] : repozytorium z kodem źródłowym aplikacji
    \item[web] : kopie źródeł elektronicznych umieszczonych w bibliografii
\end{description}

Miejsce budowania modeli i wizualizacje zawarte są w katalogu \textbf{notebooks} w postaci notatników \foreign{Jupyter}. Są to pliki z rozszerzeniem \textbf{.ipynb}. Wykonanie ich wymaga uruchomienia serwera \foreign{Jupyter}, pomocne w tym mogą być skrypty \foreign{runInDocker.sh} i \foreign{runJupyter.sh}. Przeglądać i edytować je można przez interfejs webowy. Skrypty zostały sporządzone dla systemu operacyjnego Linuks, lecz \foreign{Jupyter} i \foreign{Python} dostępne są na wielu platformach, więc uruchomienie programów powinno być możliwe również na innych platformach.

